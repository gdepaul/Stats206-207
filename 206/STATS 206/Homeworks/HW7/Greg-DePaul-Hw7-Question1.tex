%%%%%%%%%%%%%%%%%%%%%%%%%%%%%%%%%%%%%%%%%%%%%%%%%%%%%%%%%%%%%%%%%%%%%%%%%%%%%%%%%%%%
% Do not alter this block (unless you're familiar with LaTeX
\documentclass{article}
\usepackage[margin=1in]{geometry} 
\usepackage{amsmath,amsthm,amssymb,amssymb,amsfonts, fancyhdr, color, comment, graphicx, environ}
\usepackage{mathrsfs}

\usepackage{xcolor}
\usepackage{mdframed}
\usepackage[shortlabels]{enumitem}
\usepackage{indentfirst}
\usepackage{hyperref}
\hypersetup{
    colorlinks=true,
    linkcolor=blue,
    filecolor=magenta,      
    urlcolor=blue,
}


\pagestyle{fancy}
\usepackage{float}
\newcommand{\aboverightarrow}[1]{\xrightarrow[]{#1}}

\newenvironment{problem}[2][Problem]
    { \begin{mdframed}[backgroundcolor=gray!20] \textbf{#1 #2} \\}
    {  \end{mdframed}}

% Define solution environment
\newenvironment{solution}
    {\textit{Solution:}}
    {}
    \newcommand{\indep}{\perp \!\!\! \perp}
    \renewcommand\qedsymbol{$\blacksquare$}
\newcommand{\norm}[1]{\left\lVert#1\right\rVert}
\newcommand{\seminorm}[1]{\left [#1\right]}
\newcommand{\ts}{\textsuperscript}
\usepackage{scalerel}[2014/03/10]
\usepackage[usestackEOL]{stackengine}
\def\avint{\,\ThisStyle{\ensurestackMath{%
  \stackinset{c}{.2\LMpt}{c}{.5\LMpt}{\SavedStyle-}{\SavedStyle\phantom{\int}}}%
  \setbox0=\hbox{$\SavedStyle\int\,$}\kern-\wd0}\int}
\def\ddashint{\,\ThisStyle{\ensurestackMath{%
  \stackinset{c}{.2\LMpt}{c}{.5\LMpt+.2\LMex}{\SavedStyle-}{%
    \stackinset{c}{.2\LMpt}{c}{.5\LMpt-.2\LMex}{\SavedStyle-}{%
      \SavedStyle\phantom{\int}}}}\setbox0=\hbox{$\SavedStyle\int\,$}\kern-\wd0}\int}

\newcommand{\skipline}{$ \ $}

\newcommand{\reals}{\mathbb R}
\newcommand{\ints}{\mathbb Z}
\newcommand{\normal}{\trianglelefteq}
\newcommand{\onormal}{\trianglerighteq}

\newcommand{\subgroup}{\leqslant}

\newcommand{\sigalg}{\mathscr A}
\newcommand{\setsequence}{ \{ E_n \}_{n=1}^{\infty} }
\newcommand{\unionsetsequence}{ \bigcup_{i=1}^{\infty}  A_i }
\newcommand{\intersectionsetsequence}{ \bigcap_{i=1}^{\infty}  A_i }
\newcommand{\measureablespace}{(X, \sigalg)}
\newcommand{\measurespace}{(X, \sigalg, \mu)}
\newcommand{\borelspace}{\mathscr{B}(X)}
\newcommand{\lebesguemeasurespace}{(X, \borelspace, \lambda)}
\newcommand{\schwartzspace}{\mathcal S(\mathbb R^n)}
\newcommand{\temperedspace}{\mathcal S'(\mathbb R^n)}

\newcommand{\measure}{\mu: \sigalg \rightarrow [0, + \infty]} 
\newcommand{\outermeasure}{\mu: \mathbb{P}(X) \rightarrow [0, + \infty]} 
\newcommand{\convergesinmeasure}{\xrightarrow[\mu]{}} 
\newcommand{\convergesinLp}{\xrightarrow[L^p]{}} 

\newcommand{\convergesinschwartz}{\xrightarrow[]{\mathcal S}} 

\renewcommand{\qed}{\quad\qedsymbol}
\setlength\parindent{0pt}

% prevent line break in inline mode
\binoppenalty=\maxdimen
\relpenalty=\maxdimen

%%%%%%%%%%%%%%%%%%%%%%%%%%%%%%%%%%%%%%%%%%%%%
%Fill in the appropriate information below
\lhead{Greg DePaul}
\rhead{Stats 206} 
\chead{\textbf{Homework 7 Due: 18 November 2022}}
%%%%%%%%%%%%%%%%%%%%%%%%%%%%%%%%%%%%%%%%%%%%%

\begin{document}

\begin{problem}{1}
Tell true or false of the following statements and briefly explain your answer.
\begin{enumerate}[(a)]
\item  To quantify a qualitative variable with three classes $C_1, C_2, C_3$, we need the following dummy variables:
$$X_1 = \begin{cases}
1 & \text{if } C_1 \\
0 & \text{otherwise}
\end{cases}
\ \ \ X_2 = \begin{cases}
1 & \text{if } C_2 \\
0 & \text{otherwise}
\end{cases} \ \ \
X_3 = \begin{cases}
1 & \text{if } C_3 \\
0 & \text{otherwise}
\end{cases} $$

\item Polynomial regression models with higher than the third power terms are preferred since they provide better approximations to the regression relation.
\item In interaction regression models, the effect of one variable depends on the value of another variable with which it appears together in a cross-product term.
\item With a qualitative variable, the best way is to fit separate regression models under each of its classes.
\end{enumerate}
\end{problem}
\begin{solution}
\begin{enumerate}[(a)]
\item False. You can have a more compact representation with two variables for example. 
\item False. Likely to overfit. 
\item True. Interaction models are used to explain how variables interact with one another's values. 
\item False. You lose a lot of information if you don't allow the variables to be able to learn from one another. 
\end{enumerate}
\end{solution}
\end{document}